\documentclass{beamer}
%\usepackage[all,arc,curve,frame,color]{xy}
%\usepackage{tkz-graph}
\usepackage{mathtools}
\usepackage{ragged2e,etoolbox}


\newenvironment{nstabbing}
  {\setlength{\topsep}{0pt}%
   \setlength{\partopsep}{0pt}%
   \tabbing}
  {\endtabbing}

\def\jump{ \quad \\ \vspace{0.5cm} \pause}
\newcommand{\nc}{\newcommand}
\nc{\pid}{\mathfrak{p} }
\nc{\dpid}{\delta_{\mathfrak{p}}}

\def\AA{{\mathbb A}}
\def\CC{{\mathbb C}}
\def\EE{{\mathcal E}}
\def\FF{{\mathcal F}}
\def\GG{{\mathcal G}}
\def\HH{{\mathcal H}}
\def\MM{{\mathcal M}}
\def\NN{{\mathbb N}}
\def\PP{{\mathbb P}}
\def\QQ{{\mathbb Q}}
\def\RR{{\mathbb R}}
\def\ZZ{{\mathbb Z}}
\def\aa{{\mathbf a}}
\def\bb{{\mathbf b}}
\def\del{\partial}
\def\kk{\Bbbk}
\def\mm{{\mathfrak m}}
\def\nn{{\mathfrak n}}
\def\pp{{\mathfrak p}}
\def\qq{{\mathfrak q}}
\def\rr{{\mathbf r}}
\def\uu{{\mathbf u}}
\def\vv{{\mathbf v}}
\def\ww{{\mathbf w}}
\def\xx{{\mathbf x}}
\def\yy{{\mathbf y}}
\def\zz{{\mathbf z}}
\newcommand{\PGL}{\textrm{PGL}}
\newcommand{\res}{\textrm{Res}}


\DeclareMathOperator{\Tail}{Tail}
\DeclareMathOperator{\Per}{Per}
\DeclareMathOperator{\PrePer}{PrePer}

\makeatletter
\def\th@mystyle{%
    \normalfont % body font
    \setbeamercolor{block title example}{bg=orange,fg=white}
    \setbeamercolor{block body example}{bg=orange!20,fg=black}
    \def\inserttheoremblockenv{exampleblock}
  }
\makeatother

\makeatletter
\def\th@thmstyle{%
    \normalfont % body font
    \setbeamercolor{block title example}{bg=blue,fg=white}
    \setbeamercolor{block body example}{bg=blue!20,fg=black}
    \def\inserttheoremblockenv{exampleblock}
  }
\makeatother

\definecolor{darkgreen}{RGB}{77,153,0}
\makeatletter
\def\th@qstnstyle{%
    \normalfont % body font
    \setbeamercolor{block title example}{bg=darkgreen,fg=white}
    \setbeamercolor{block body example}{bg=green!20,fg=black}
    \def\inserttheoremblockenv{exampleblock}
  }
\makeatother


\theoremstyle{thmstyle}
\newtheorem*{mythm}{Theorem}
\newtheorem*{mydef}{Definition}
\newtheorem*{myproof}{Proof}

\theoremstyle{mystyle}
\newtheorem*{remark}{Remark}
\newtheorem*{conjecture}{Conjecture}
\newtheorem*{myexample}{Example}
\newtheorem*{mycor}{Corollary}
\newtheorem*{mylemma}{Lemma}

\theoremstyle{qstnstyle}
\newtheorem*{question}{Question}

\usepackage{remreset}% tiny package containing just the \@removefromreset command
\makeatletter
\@removefromreset{subsection}{section}
\makeatother
\setcounter{subsection}{1}

\newcommand\Wider[2][3em]{%
\makebox[\linewidth][c]{%
  \begin{minipage}{\dimexpr\textwidth+#1\relax}
  \raggedright#2
  \end{minipage}%
  }%
}

\mode<presentation>{\usetheme{CambridgeUS}\usecolortheme{dolphin}} 
%\setbeamertemplate{navigation symbols}{}
\setbeamertemplate{blocks}[rounded][shadow=false]


\title[Formally Real Fields.]{Formally Real Fields}
%\subtitle{long subtitle}
\author[Sebastian Troncoso]{Sebastian Troncoso}
\institute[BSC]{Birmingham-Southern College}
%\titlegraphic{\includegraphics[height=1.5cm]{../images/normale_pisa.png}}
\date[ \today]{ \today \\ \vspace{1cm} }


%\AtBeginSection[]{} % for optional outline or other recurrent slide
\AtBeginSection{\frame{\sectionpage}}
\begin{document}

\begin{frame}
\titlepage
\end{frame}

\begin{frame}[t]
\frametitle{Fields}
First, let's recall the definition of a field. \pause
\begin{mydef}
A \textbf{field} $F$ is a set, along with two operations defined on the set: an addition and a multiplication such that $(F,+)$ and $(F^{*},\cdot)$ are abelian groups and distributivity law holds in $F$. 
\end{mydef}

\pause

Classical examples are: 
\begin{itemize}
\item  $\QQ$ 

\item $\RR$ 

\item $\CC$
\pause

\item $\QQ(\sqrt{2})=\{x\in \RR \mid x=a+b\sqrt{2} \quad a,b\in \QQ \}$
\end{itemize}
\end{frame}

\begin{frame}[t]
\frametitle{Definition}
Informally, a formally real field (also called ordered field) is a field with a linear order such that the operations of the fields are preserved \emph{i.e.}
$$x\leq y \Longrightarrow x+z\leq y+z $$
$$x\leq y \mbox{ and } 0\leq z \Longrightarrow xz\leq yz.$$

\pause
\begin{mydef}
Let $F$ be a field. An \textbf{ordering} $\leq$ of $F$ is a binary relation satisfying
\begin{enumerate}
\item $a\leq a$
\item $a\leq b, b\leq c \Longrightarrow a\leq c$
\item $a\leq b, b\leq a \Longrightarrow a=b$
\item $a\leq b$ or $b\leq a$
\item $a\leq b \Longrightarrow a+c \leq b+c$
\item $0\leq a, 0\leq b \Longrightarrow 0\leq ab$
\end{enumerate}
\end{mydef}
\end{frame}

\begin{frame}[t]
\frametitle{Example}
\begin{myexample}
$\QQ$ and $\RR$ are ordered fields with their usual orderings. 
\end{myexample}

\jump

\begin{myexample}
Any field $F$ such that $\QQ \subset F \subset \RR$ with the usual ordering is a ordered field.\jump For instance, $\QQ(\sqrt{2})$ 
\end{myexample}


\end{frame}

\begin{frame}[t]
\frametitle{Positive Numbers}
Let $F$ be a ordered field and consider the set of positive numbers, $P=\{a\in F \mid 0 \leq a \}$ then $P$ satisfies the following axioms
\jump
\begin{enumerate}
\item $P+P\subset P$
\item $P \cdot P \subset P $
\item $P \cap -P = \{0 \} $
\item $P \cup -P=F$
\end{enumerate}
where $-P=\{-a\in F \mid a\in P \}=\{b\in F \mid 0 \geq b \}$.

\jump
If $a\geq 0 $ then \pause  
$$a+(-a) \geq 0+(-a)$$
$$0\geq -a $$
\end{frame}

\begin{frame}[t]
\frametitle{Positive Cone}

\begin{mydef}
Let $F$ be a field and $P \subset F$. We say that $P$ is a \textbf{positive cone} if such that $P$ satisfies the following axioms
\begin{enumerate}
\item $P+P\subset P$
\item $P \cdot P \subset P $
\item $P \cap -P = \{0 \} $
\item $P \cup -P=F$
\end{enumerate}
where $-P=\{-a\in F \mid a\in P \}$. 
\end{mydef}

\pause
On the previous slide, we saw that given an order we can naturally construct a positive cone given by the positive numbers. 
\jump 
Conversely, if $P$ is a positive cone then 
$$a \leq b \iff b-a \in P $$
defines an ordering on $F$ such that $P$ is its positive cone. 

\jump 
Hence, the concepts of positive cones and ordering of a field are equivalent.
\end{frame}

\begin{frame}[t]
\frametitle{Lemma: Squares are positive}
\begin{mylemma}
Let $F$ be a ordered field. Then 
\begin{enumerate}
\item $0 \leq a \Longrightarrow -a \leq 0 $ for every $a\in F$. 

\item $0 \leq 1 $  and $-1 \leq 0 $.

\item $0\leq a^2$ for every $a\in F$.
\end{enumerate}
\end{mylemma}
\pause
\begin{myproof}
\begin{enumerate}
\item DONE.
\pause \item If $1 \leq 0$ then $0 \leq -1$. Hence $0 \leq (-1)^2=1$ which is a contradiction since the only element which is positive and negative at the same time is 0.
\pause \item If $0\leq a$ then $0\leq a^2$ by definition of order. If $a \leq 0$ then $0\leq -a$. Hence, $0\leq (-a)^2=a^2$ by definition of order.
\end{enumerate}

\end{myproof}
\end{frame}

\begin{frame}[t]
\frametitle{Nonexample}
\begin{myexample}
$\CC$ is \textbf{NOT} an ordered field because $i^2=-1$.
\end{myexample}
\jump
\begin{myexample}
From a previous talk we know that $\mathbb{Z}_p$ is a field when $p$ is a prime number. $\mathbb{Z}_p$ is \textbf{NOT} an ordered field because 
\begin{equation*}
\begin{split}
p-1 & = 1+1+\ldots+1 \geq 0 \\
& = -1 \leq 0
\end{split}
\end{equation*}
\end{myexample}

\end{frame}



\begin{frame}
\begin{center}
{\Huge{THANK YOU}}
\end{center}


\end{frame}


\end{document}